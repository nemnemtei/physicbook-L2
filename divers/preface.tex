\chapter*{Avant-propos}

\addcontentsline{toc}{chapter}{Avant-propos} 

Si l'apprentissage et la curiosité sont parmis les plus grandes activités de l'humain au cours de sa vie, il est nécessaire d'appuyer sur le point que, sans grands professeurs et grands chercheurs, aujourd'hui, notre civilisation et notre culture ne serait pas ce qu'elle est. Une société ne peut fonctionner sans ses écoles, ses collèges, ses lycées et ses universités. La connaissance que l'on transmet perpétuellement est une richesse infinie, dont la duplication est sans limites et ainsi sans pertes. \\

Le travail d'un étudiant universitaire est d'autant plus difficile car il doit négocier entre travail en autonomie, synthèse, organisation et productivité. Alors voici ici un petit manuel d'apprentissage regroupant ce que j'ai personnellement tenu à synthétiser, organiser et produire dans le but de compiler les connaissances acquises au cours d'un semestre en licence de physique. Ceci afin de pouvoir, en plus d'assimiler une quantité folle d'informations, transmettre et partager un savoir dont l'accès, s'il n'est pas difficile de premier abord, peut parfois être plus complexe à son assimilation.\\

Je tiens à relever que l'intégralité de ce document est issu des enseignements, pour la thermodynamique du professeur Aziz \textsc{Ghoufi} et pour l'électromagnétisme du professeur Christophe \textsc{Cappe}.\\

Je tiens tout personnellement à remercier les professeurs qui au quotidien travaillent à transmettre un savoir, une quantité importante de connaissances, dont  chacun d'entre nous a besoin dans des buts diverses et variés. Je remercie particulièrement le professeur Sergio \textsc{Di Matteo} pour m'avoir appris à rédiger des documents sur \LaTeX en particulier ceux sur les synthèses de cours, monsieur Franck \textsc{Paysant} pour sa passion des schémas et du discours oral et monsieur Alexandre \textsc{Travers} pour m'avoir montré la voie des sciences et de l'apprentissage, je ne sais où je serais aujourd'hui sans ce dernier.\\

Enfin, je tiens à remercier mes camarades de promotions dont la relecture attentive de ce document a permis d'améliorer son contenu et d'en éliminer les quelques coquilles qui s'y cachait, j'ai nommé Ewen \textsc{Le Ster}, Thomas \textsc{Jamin}, Raphaël \textsc{Boisbourdin} et Huo \textsc{Vallée}.\\

Il ne me reste plus qu'à souhaiter à celui qui lit ceci une réussite la plus exceptionnelle soit elle, sans oublier que, dans toute situation, le travail est la valeur qui prime et que la réussite parvient toujours à celui qui sait s'impliquer suffisament dans ses études.\\

Nemetan \textsc{TEIXEIRA RUA}
